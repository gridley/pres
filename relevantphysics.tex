\begin{frame}
  \frametitle{MSR Governing Equations}
      \textbf{Neutrons'} changing concentrations can be described approximately with coupled diffusion equations:
      %\begin{equation}
      %\begin{aligned}
      %    \frac{1}{v(E)} \pd{\psi}{t} + \hat{\Omega}\cdot \nabla \psi + \textcolor{red}{\Sigma_t} \psi =& \frac{\chi_p(E)}{4 \pi} \int_{0}^{\inf} \nu_p(E') \textcolor{red}{\Sigma_f} \textcolor{purple}{\phi} dE' + \sum_i \frac{\chi_d(E)}{4 \pi} \lambda _i C_i + \\
      %                                                                                                  & \int_{4\pi}  \int_{0}^{\inf} \textcolor{red}{\Sigma_s(E'\rightarrow E,\hat{\Omega}'\rightarrow \hat{\Omega})} \psi dE' d\hat{\Omega}\\
      %\end{aligned}
      %\end{equation}
		%% \frac{1}{v_g}\frac{\partial \textcolor{purple}{\phi}_g}{\partial t} - \nabla \cdot D_g \nabla \phi_g
		%% + \Sigma_g^r \textcolor{purple}{\phi}_g = \sum_{g \ne g'}^G \Sigma_{g'\rightarrow g}^s \phi_{g'} + \chi_g^p \sum_{g' = 1}^G (1 - \beta)
		%% \nu \Sigma_{g'}^f \textcolor{purple}{\phi}_{g'}
\begin{equation}
				\frac{1}{v_g}\frac{\partial \textcolor{purple}{\phi}_g}{\partial t}   = \nabla \cdot D_g
				\nabla \textcolor{purple}{\phi}_g +
				\sum_{g \ne g'}^G
                \textcolor{red}{\Sigma_{g'\rightarrow g}^s} \textcolor{purple}{\phi}_{g'} + \chi_g^p \sum_{g' = 1}^G (1 -
                \beta) \nu \textcolor{red}{\Sigma_{g'}^f} \textcolor{purple}{\phi}_{g'} + \chi_g^d \sum_i^I \lambda_i \textcolor{green}{C_i} - \textcolor{red}{\Sigma_g^r} \textcolor{purple}{\phi_g}
		\label{eq:neutrons}
\end{equation}
    \textbf{Delayed neutron precursors} are freshly split atoms that drift in the salt current, later emit neutrons.

    \begin{equation}
        \frac{\partial \textcolor{green}{C_i}}{\partial t} = \sum_{g'= 1}^G \beta_i \nu
        \textcolor{red}{\Sigma_{g'}^f} \textcolor{purple}{\phi}_{g'} - \lambda_i \textcolor{green}{C_i} - \frac{\partial}{\partial z} u
        \textcolor{green}{C_i} \label{eq:precursors}
    \end{equation}                                                                 				%where

    \textbf{Heat} changes coefficients in other equations.
\begin{equation}
    \rho_fc_{p,f}\frac{\partial \textcolor{red}{T_f}}{\partial t} + \nabla\cdot\left(\rho_f
    c_{p,f} \vec{u}\cdot \textcolor{red}{T_f} -k_f\nabla \textcolor{red}{T_f}\right) = \sum_g \textcolor{purple}{\phi_g} \textcolor{red}{\Sigma_{f,g}} E_{f,g}
  \label{eq:fuel_temp}                                                        				%$v_g    $    = \mbox{speed of neutrons in group g} \\
\end{equation}                                                                				%$\textcolor{purple}{\phi}_g $    = \mbox{flux of neutrons in group g} \\

\textcolor{red}{Red values} change with temperature. \textcolor{green}{Green} refers to delated neutron precursors. \textcolor{purple}{Purple} indicates neutron flux.
				%$t      $    = \mbox{time} \\
				%$D_g    $    = \mbox{Diffusion coefficient for neutrons in group g} \\
				%$\Sigma_g^r$ = \mbox{macroscopic cross-section for removal of neutrons
				%from group g} \\
				%$\Sigma_{g'\rightarrow g}^s$ = \mbox{macroscopic cross-section of
				%scattering from g' to g} \\
				%$\chi_g^p$   = \mbox{prompt fission spectrum, neutrons in group g} \\
				%$G$          = \mbox{number of discrete groups, g} \\
				%$\nu$        = \mbox{number of neutrons produced per fission} \\
				%$\Sigma_g^f$ = \mbox{macroscopic cross section for fission due to neutrons in group g} \\
				%$\chi_g^d$   = \mbox{delayed fission spectrum, neutrons in group g} \\
				%$I $         = \mbox{number of delayed neutron precursor groups} \\
				%$\beta $     = \mbox{delayed neutron fraction}\\
				%$\lambda_i $ = \mbox{average decay constant of delayed neutron precursors
				%in precursor group i} \\
				%$C_i $       = \mbox{concentration of delayed neutron precursors in precursor
				%group i} .
\end{frame}

%\begin{frame}
%    \frametitle{MSRs: An Intrinsically Coupled System II}
%    \textbf{Delayed neutron precursors} are products of freshly split uranium that emit a new neutron after a delay.
%    \textit{critical} to reactor control; they shift power change timescales from picoseconds to seconds despite
%    only being a few tenths of a percent of emitted neutrons.
%
%	\begin{equation}
%			\frac{\partial C_i}{\partial t} = \sum_{g'= 1}^G \beta_i \nu
%			\Sigma_{g'}^f \textcolor{purple}{\phi}_{g'} - \lambda_i C_i - \frac{\partial}{\partial z} u
%			C_i \label{eq:precursors}
%	\end{equation}
%
%    % note, flow is primarily only the +z direction
%
%    \textbf{Heat and temperature} affect the coefficients in Equation \ref{eq:neutrons} significantly. Energy conservation must be solved:
%
%\begin{equation}
%        \rho_fc_{p,f}\frac{\partial T_f}{\partial t} + \nabla\cdot\left(\rho_f
%        c_{p,f} \vec{u}\cdot T_f -k_f\nabla T_f\right) =  Q_f
%  \label{eq:fuel_temp}
%\end{equation}
%
%\begin{tabular}{cc}
%  $\rho_f $ &= \mbox{density of fuel salt}\\
%  $c_{p,f}$ &= \mbox{specific heat capacity of fuel salt}\\
%  $T_f    $ &= \mbox{temperature of fuel salt}\\
%  $\vec{u}$ &= \mbox{velocity of fuel salt}\\
%  $k_f    $ &= \mbox{thermal conductivity of fuel salt}\\
%  $Q_f    $ &= \mbox{source term}\\
%\end{tabular}
%
%\end{frame}
